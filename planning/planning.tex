\chapter{Project Planning}
\section{Risk Management}
As a matter of practicality, it is useful to have some idea of what risks
and potential delays are involved in a project such as this. Rather than
take the usual approach of approximations and guesswork, I decided to
approach the assessment of risks quantitatively. In the risk assessment
table that follows, I have given for each risk a \emph{likelihood} -- a
probability between 0 and 1 inclusive, and an \emph{impact} given in the
amount of days of delay the risk may induce. This gives, for each risk, an
expected delay in days by taking the product of the two values.

\subsection{Data Loss (implementation)}

\textbf{Data Loss of implementation} would be an extremely frustrating
setback. This could happen easily through physical damage to its storage
media. On the Gantt chart the implementation time is estimated as \textbf{3
weeks} so that is the figure I present for the impact: 21 days.

To estimate the probability, I consider the three major failure modes:
failure of storage medium, failure of the computer to which it is attached,
and failure of the building containing it.

Toshiba state the mean time between failure of their hard disks as
600,000 hours, which is 25,000 days\cite{toshibaMTBF}. We can therefore
infer that the number of failures in any particular day follows a
Poisson distribution: $\operatorname{Po}(\frac{1}{25000})$.

I would estimate the mean lifetime of an Apple MacBook Pro -- the machine I
am using -- at 4 years. I state this, unfortunately, without evidence. This
figure gives a MTBF of 1,461 days, so by a similar mechanism we infer the
Poisson distribution $\operatorname{Po}(\frac{1}{1461})$.

The University of Southampton has burnt down two buildings within
the last 10 years\cite{fireRecord, fireRecord2}. This corresponds
to a MTBF of 1,825.25 days, giving rise to the third Poisson
distribution $\operatorname{Po}(\frac{1}{1825.25})$.

From the sum of these we can conclude that the number of catastrophic
instances of data loss per day follows $\operatorname{Po}(\frac{339291761}
{266669025000})$. By the usual simple tools of probability and the
probability mass function of the Poisson distribution, we can
conclude that the probability of a nonzero number of failures over
70 days (10 weeks between the start of implementation and the end
of the project) is approximately \textbf{8.52\%}.

\subsection{Data Loss (report)}

\textbf{Data Loss of the final report} would be equally frustrating to
losing the implementation, although luckily for the consideration of its
risk I am given a moment of respite from the frustration by being able to
reuse most of the data loss calculations from the consideration of the
implementation. The impact I would estimate at \textbf{56 days} since this
is the time put aside to the interim report, draft report and final report
in the Gantt chart.

The number of catastrastrophic failures per day remains the same as before,
following $\operatorname{Po}(\frac{339291761}{266669025000})$. What changes
is the number of days: the interim report is started 8 weeks before the
implementation, giving an additional 56 days for a total at-risk period of
126 days, from which we can conclude a probability of no failures of that period
of $85.2\%$, or equivalently a \textbf{14.8\%} probability of failure.

\todo{Write up something about mitigation here.}

\subsection{Illness or Injury}

Personal illness or injury is always a risk in long-term projects.
For this particular risk I will not (for obvious privacy reasons)
go into in-depth calculations but will simply state an expected
delay of \textbf{4 weeks} and refer readers to FPSE's records for
further information.

\begin{figure}
\begin{tabu} to 0.9\linewidth { X[7,r] | X[c] | X[c] | X[c] }
  \textbf{Risk} & \rot{\textbf{Likelihood}} &
    \rot{\textbf{Impact (days)}} & \rot{\textbf{Expected delay (days)}} \\
    \hline
    Data Loss (implementation) & 8.5\%
                               & 21
                               & 1.8 \\
    Data Loss (report) & 14.8\%
                       & 56
                       & 8.3 \\
    Health/Injury & --
                  & --
                  & 28
\end{tabu}
\caption{Summary of risks.}
\label{fig:risks}
\end{figure}

% \begin{tabu} to 0.9\linewidth { X[3,r] | X[7,l] | X[c] | X[c] | X[7,l] }
% \textbf{Risk} & \textbf{Description} & \rot{\textbf{Risk level (1--5)}} & \rot{\textbf{Likelihood (1--5)}} & \textbf{Mitigation} \\ \hline
% Data Loss &
%   Loss of data through corruption, or loss of hard disks. &
%   4 & 4 &
%   Periodic backups. Duplicates of data held in ECS with ECS's backup
%   systems\footnote{ECS's fire safety record notwithstanding\cite{fireRecord}.} and on GitHub (as described in Section~\ref{sub:github}). \\
% Personal Injury &
%   Various forms of personal incapacitation inhibiting progress. &
%   3 & 2 &
%   Staged planning, first drafted completed six weeks in advance. \\
% \end{tabu}
\begin{landscape}
\section{Gantt Chart}

\todo{Include the contingencies.}

\resizebox{\linewidth}{!}{
\begin{ganttchart}[y unit chart=0.5cm,
                   hgrid=true,
                   vgrid={dotted}]{18}
  \gantttitle{Aug}{2}
  \gantttitle{Sep}{2}
  \gantttitle{Oct}{2}
  \gantttitle{Nov}{2}
  \gantttitle{Dec}{2}
  \gantttitle{Jan}{2}
  \gantttitle{Feb}{2}
  \gantttitle{Mar}{2}
  \gantttitle{Apr}{2} \\
  \ganttbar{Initial Research}{1}{8} \\ % elem0
  \ganttbar{Project Brief}{4}{4} \\ % elem1
  \ganttbar{Interim Report}{8}{9} \\ % elem2
  \ganttbar{Prototyping}{9}{11} \\ % elem3
  \ganttbar{Implementation}{12}{14} \\ % elem4
  \ganttbar{Draft Report}{13}{15} \\ % elem5
  \ganttbar{Final Report}{16}{18} \\ % elem6
  \ganttbar{Supervision}{1}{18} % elem7
  \ganttlink{elem0}{elem3}
  \ganttlink{elem3}{elem4}
  \ganttlink{elem4}{elem6}
  \ganttlink{elem5}{elem6}
  \ganttlink{elem1}{elem2}
  \ganttlink{elem2}{elem5}
\end{ganttchart}
}
\end{landscape}

